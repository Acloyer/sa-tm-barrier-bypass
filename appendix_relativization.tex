% appendix_relativization.tex
\section{Relativization Barrier Details}

Classical relativizing lower-bound techniques
(Baker–Gill–Solovay, circuit simulations à la Bennett, 
and the linear-speed-up argument) assume that
\emph{any} black-box call to oracle~$\OO$ can be reproduced by
a universal machine that merely intercepts the query string.
SA-TMs break this assumption, because a query may depend on
\emph{bits of the machine description that are
outside the radius $\tau(n)=\Theta(\log n)$ of any external simulator}. 
Below we formalise this obstruction.

\begin{theorem}\label{thm:nonrel}
Let $U$ be any deterministic oracle TM that simulates
every SA-TM $M$ for at most $p(|x|)$ overhead
and issues each oracle question \emph{verbatim}.
Then $p(n)$ must be super-polynomial.
\end{theorem}

\begin{proof}
Fix $n$ and consider the following SA-TM $M_n$ on empty input~$\epsilon$.
\begin{enumerate}
  \item Read its own code tape within radius $\tau(n)$,
        thereby learning the first $\Theta(\log n)$ bits of its Gödel index~$i_n$.
  \item Construct string $x_n$ of length $n$ 
        that explicitly records those bits and pads by~$0$’s.
  \item Query the oracle at $q=\langle\Diag,i_n,x_n\rangle$ and output the reply.
\end{enumerate}
By Lemma~\ref{lem:nocirc}, $|q|>\tau(n)$,
so \emph{none} of the bits inspected on the code tape suffices
to reconstruct the full $q$.
Any ordinary TM~$U$ that wishes to simulate step 3
must explicitly \emph{output} $q$ on its own oracle channel.
Hence $U$ must embed all $\Theta(n)$ undocumented bits of $i_n$
into its work tape, violating the assumed polynomial overhead.
Formally, otherwise we would compress $i_n$ to $O(\log n)$ bits,
contradicting the Kolmogorov-incompressibility of a random index.
\end{proof}

\begin{corollary}
The separation $\classP_{\SA}^\OO\!\neq\!\classNP_{\SA}^\OO$
of Section~\ref{sec:oracle} is \emph{non-relativizing}:
there is no black-box proof that resolves $P$ vs $NP$
in the SA-model uniformly for \emph{all} oracles.
\end{corollary}

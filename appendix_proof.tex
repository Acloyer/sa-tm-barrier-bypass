% Copyright (c) 2025 Rafig Huseynzade. All Rights Reserved.
% Licensed under CC BY-NC-ND 4.0
% Original work - do not copy without attribution
\section{Proof-Complexity Lower Bound}

Recall $\tau_n$
(Definition~\ref{thm:proof}):  
“no SA-TM of size $\le n$ with pattern $\Diag_n$
accepts its own code”.

\subsection{Upper bound: poly-size SA-proofs}

\begin{lemma}\label{lem:sa-proof}
There exists an SA-Frege proof of $\tau_n$
of size $O(n^2)$.
\end{lemma}

\begin{proof}
The proof carries out the diagonal construction
\emph{inside} the proof system:
each derivation line is either
(i) a local copy of one transition
(read via \texttt{TRANS}), or
(ii) an arithmetic equality justifying the
padding length $|x|>4\log n$.
Since every INT query reads $\le\tau(n)=O(\log n)$ bits,
encoding one line takes $O(\log n)$ symbols,
hence total size $O(n^2)$.
\end{proof}

\subsection{Lower bound against Frege}

\paragraph{Outline.}
We interpolate between SA-tautologies and
the Razborov–Smolensky pigeonhole principle (PHP),
whose Frege size lower bound is $n^{\Omega(\log n)}$.

\begin{definition}[Gadget encoding]\label{def:gadget}
Map each pigeon $p\in[n+1]$ to a unique
pattern $g(p)\in\{0,1\}^{m}$ 
whose first $\Theta(\log n)$ bits equal~$p$.
The SA-pattern $\Diag_n$ contains every $g(p)$
inside its self-reference query.
\end{definition}

\begin{lemma}[Feasible interpolation]\label{lem:interpolation}
Any Frege proof of $\tau_n$ of size $s$
yields a Boolean circuit of size $s^{O(1)}$
separating
\(\textit{PHP}_{n+1\to n}\) from its negation.
\end{lemma}

\begin{proof}
Standard Krajíček–Razborov interpolation:
variables corresponding to
$g(p)$ act as \emph{selector} wires.
Since $\tau_n$ is of the form
$\bigvee_p C_p$ with each clause $C_p$
mentioning \emph{disjoint} symbol sets,
the circuit splits into $s^{O(1)}$ monotone gates.
\end{proof}

\begin{theorem}[Frege lower bound]\label{thm:frege-lb}
Every Frege proof of $\tau_n$ has size
$n^{\Omega(\log n)}$.
\end{theorem}

\begin{proof}
If a shorter Frege proof existed,
Lemma~\ref{lem:interpolation} would give
a circuit contradicting the known
Razborov \cite{Raz87} lower bound
$\textit{size}\!>\!n^{\Omega(\log n)}$
for monotone $\textit{PHP}$ circuits.
\end{proof}

\paragraph{Remark.}
The separation exploits the \emph{local-code} feature:
Frege cannot efficiently encode the many
independent address bits hidden in $\Diag_n$,
whereas SA-Frege gains them at $O(\log n)$ cost
via \texttt{CODE\_TAPE}.

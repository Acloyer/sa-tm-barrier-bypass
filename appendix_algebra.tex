% Copyright (c) 2025 Rafig Huseynzade. All Rights Reserved.
% Licensed under CC BY-NC-ND 4.0
% Original work - do not copy without attribution
\section{Algebraization: Exponential Degree Lower Bound}

We restate Theorem~\ref{thm:algebra}:

\begin{theorem*}
Let $m$ be the bit-length of an SA-query
$q=\langle\Diag,i,x\rangle$.  
Any polynomial $P:\F^m\!\to\F$ that agrees with
the Boolean function $f_m$ on $\{0,1\}^m$
must have $\deg P \ge 2^{\Omega(m)}$.
\end{theorem*}

\subsection{Derivative method}

Write $\Delta_{e_j}P(z)=P(z+e_j)-P(z)$.
For $k$-tuple $S\subseteq[m]$ define
$\Delta_S P = \Delta_{e_{j_1}}\cdots\Delta_{e_{j_k}}P$,
$k=|S|$.

\begin{lemma}\label{lem:support}
For every $z\in\{0,1\}^m$ the value $f_m(z)=1$
iff $z$ encodes a self-diagonalising query.
Hamming balls of radius $\le m/4$ around those $z$
are \emph{disjoint}.
\end{lemma}

\begin{proof}
Each such $z$ embeds a minimal Gödel index $i$
and padded input~$x$;
changing $\le m/4$ coordinates
cannot transform it into another valid encoding
due to prefix-free coding of $i$.
\end{proof}

\begin{lemma}\label{lem:derivative}
If $\deg P < 2^{m/4}$, 
then $\Delta_S P \equiv 0$ for all $|S|=2^{m/4}$
by basic polynomial calculus.
\end{lemma}

Choose $S$ hitting one bit in each disjoint
ball of Lemma~\ref{lem:support}.
$f_m$ restricted to that $S$ remains
\emph{non-zero}, hence
$\Delta_S P$ must be non-zero on $\{0,1\}^{m-|S|}$,
contradiction.

\begin{proof}[Completion of proof]
Set $k=2^{m/4}$; any agreeing polynomial
must have degree $\ge k$, i.e.\ 
$2^{\Omega(m)}$.
\end{proof}
